\documentclass[11pt]{article}

\usepackage{sectsty}
\usepackage{graphicx}

% Margins
\topmargin=-0.45in
\evensidemargin=0in
\oddsidemargin=0in
\textwidth=6.5in
\textheight=9.0in
\headsep=0.25in

\title{ Title}
\author{ Author }
\date{\today}

\begin{document}
\maketitle	

% Optional TOC
% \tableofcontents
% \pagebreak

%--Paper--

\section{Functionalities}
\subsection{Operator functionalities}
All the functionalities of the operators are implemented in the file lottery\_manager.js.
The lottery can be managed from the page lottery\_manager.html.
Here the interface will change dinamically, according to the current state of the lottery:
\begin{itemize}

\item If a round is not yet started, the button "Start Round" will be enabled.
\item If there is an active round, then the interface will show the number of remaining blocks to pass.
\item If the round is not active, then a "Draw numbers" button will appear.
\begin{figure}
\includegraphics[width=\textwidth]{lottery.png}

\end{figure}
\end{itemize}
\subsubsection*{Creating a new lottery}
With the button "Create Lottery" is possible to create a new lottery. However, in order to perform this operation,
the previous round must be finished.

\subsubsection*{Closing the lottery}
With the button "Close Lottery" is possible to close the current lottery. Once a lottery has been deactivated,
in the page it will be shown only option to create a new lottery.

\subsection{Tickets}
The page buy.html is used to buy new tickets and show those previously bought. If an error occur when the user tries to buy a
ticket (e.g numbers constraint not respected, Round not active, etc...), a red alert will appear above to flag the error.
\begin{figure}[h!]
\centering
\includegraphics[width=0.8\textwidth]{ticket.png}
\caption{Buy tickets}
\label{fig:buy}
\end{figure}

\subsection{NFTs}
\subsubsection*{Prizes}
The page prizes.html is used to show the prizes of the current round.
\begin{figure}[h!]
\centering
\includegraphics[width=0.8\textwidth]{prizes.png}
\caption{Prizes}
\label{fig:prizes}
\end{figure}

\subsubsection*{NFTs won}
The page nft.html is used to show the NFTs won by the user. It uses the
same format as the page for prizes, as previously showed.
%--/Paper--
\section{Events}
When an events is triggered, a toast message will be shown on the bottom right corner of the screen
with the related event.
the events managed by the lottery are:
\begin{itemize}
\item The creation of a new lottery
\item The start of a new round
\item The extraction of the numbers
\item The winning of an NFT
\end{itemize}
\begin{figure}[h!]
\centering
\includegraphics[width=0.8\textwidth]{event_numbers.png}
\caption{An example of toast to represent an event}
\label{fig:events}

\end{figure}


\section{Implementation details}
\subsection{Lottery Factory}
I created a new contract to manage the creation and migration of new lotteries from the
UI. It is called LotteryFactory. It has two main functions: CreateLottery and getLotteryAddress.
The first one is used to create a new lottery and the second one is used to get the address of the
the currently active lottery, which will be used to instantiate a TruffleContract in the JavaScript
code to interact with the lottery. The Lottery factory has to be deployed with
the Truffle migration script, which it will also create the first lottery.
\subsection{Core.js}
In the file core.js, I implemented the functions to initialize all the lottery machinery.
The function init returns an object with all the necessary object instantiated to interact
with the blockchain. The file is imported in every other necessary file, so I could avoid
code duplication.
\subsection{GUI}
For the implementation of the GUI, I used the Bootstrap CSS framework.
\subsection{Test}
To facilitate the testing of the application, I automated the buy of the tickets in the migration script
(1\_initial\_migration.js). Then the operator can immediately extract the numbers and the prizes are given. One
can see the NFT won in the nft.html page. This avoid the burden of buying tickets
every time from scratch to test the lottery. However, for the delivery of the project
I commented those lines, so the lottery will start from scratch.

\end{document}
